\documentclass[12pt]{article}
\usepackage[pdftex]{graphicx}
\usepackage[T1]{fontenc}
\usepackage[polish]{babel}
\usepackage[utf8]{inputenc}
\usepackage{listings}
\usepackage{xcolor}

\usepackage[a4paper, left=2.5cm, right=2.5cm, top=3.5cm, bottom=3.5cm, headsep=1.2cm]{geometry}

\linespread{1.3}

\begin{document}
	
	\begin{titlepage}
		\centering
		{\scshape\LARGE Politechnika Wrocławska \par}
		{\scshape\Large Katedra Informatyki Technicznej\par}
		
		\vspace{1.5cm}
		{\scshape\Large Urządzenia Peryferyjne \par}
		\vspace{1.5cm}
		{\scshape\Huge GPS}
		\vspace{1.5cm}
	
		\vspace{2cm}
		{\Large\itshape Magdalena Biernat\par}
		{\Large\itshape Michał Bojzan \par}
		\vfill\flushleft\large
	
		
		\normalsize	\centering	\vspace{3cm}
		Prowadzący\par
		dr inż. Jan Nikodem 

		\vfill
		{\large \today\par}
	\end{titlepage}

	\newpage
\section{Wstęp}
\subsection{Co to jest GPS?}
Global Positioning System (GPS) – właściwie GPS–NAVSTAR (ang. Global Positioning System – Navigation Signal Timing and Ranging) – system nawigacji satelitarnej, stworzony przez Departament Obrony Stanów Zjednoczonych, obejmujący swoim zasięgiem całą kulę ziemską. System składa się z trzech segmentów: segmentu kosmicznego – 31 satelitów orbitujących wokół Ziemi na średniej orbicie okołoziemskiej; segmentu naziemnego – stacji kontrolnych i monitorujących na Ziemi oraz segmentu użytkownika – odbiorników sygnału. Zadaniem systemu jest dostarczenie użytkownikowi informacji o jego położeniu oraz ułatwienie nawigacji po terenie.
\subsection{Sposób działania GPSa}
Działanie polega na pomiarze czasu dotarcia sygnału radiowego z satelitów do odbiornika. Znając prędkość fali elektromagnetycznej oraz znając dokładny czas wysłania danego sygnału można obliczyć odległość odbiornika od satelitów. Sygnał GPS zawiera w sobie informację o układzie satelitów na niebie (tzw. almanach) oraz informację o ich teoretycznej drodze oraz odchyleń od niej (tzw. efemeryda). Odbiornik GPS w pierwszej fazie aktualizuje te informacje w swojej pamięci oraz wykorzystuje w dalszej części do ustalenia swojej odległości od poszczególnych satelitów, dla których odbiornik jest w zasięgu. Wykonując przestrzenne liniowe wcięcie wstecz mikroprocesor odbiornika może obliczyć pozycję geograficzną (długość, szerokość geograficzną oraz wysokość elipsoidalną) i następnie podać ją w wybranym układzie odniesienia – standardowo jest to WGS 84, a także aktualny czas GPS z bardzo dużą dokładnością.
Sygnał dociera do użytkownika na dwóch częstotliwościach nośnych L1 = 1575,42 MHz (długość fali 19,029 cm) i L2 = 1227,6 MHz (długość fali 24,421 cm). Porównanie różnicy faz obu sygnałów pozwala na dokładne wyznaczenie czasu propagacji, który ulega nieznacznym wahaniom w wyniku zmiennego wpływu jonosfery, jednak nie w stopniu uniemożliwiającym określenie współrzędnych. Użytkownicy cywilni przybliżoną poprawkę jonosferyczną otrzymują w depeszy nawigacyjnej lub dzięki systemowi DGPS.
Identyfikacja satelitów oparta jest na metodzie podziału kodu CDMA (Code Division Multiple Access). Oznacza to, że wszystkie satelity emitują na tych samych częstotliwościach, ale sygnały są modulowane różnymi kodami.
Odbiór sygnału bez zastosowania anten parabolicznych, które w tym przypadku są bezużyteczne ze względu na ich kierunkowość, wymaga zaawansowanych technik oddzielania sygnału od szumu i przetwarzania sygnału. Satelity są w ciągłym ruchu; wyznaczenie pozycji odbiornika na podstawie pomiaru tzw. pseudoodległości od kilku satelitów jest również złożonym zadaniem, wymagającym m.in. uwzględnienia spowolnienia upływu czasu wynikającego ze zjawiska dylatacji czasu.
\subsection{Rodzaje odbiorników}
Dla jednoczesnego odbioru sygnału z kilku satelitów lub sygnału o dwóch częstotliwościach z jednego satelity, stosuje się odbiorniki dwóch rodzajów:
\begin{itemize}
\item multi-channel (wielokanałowy) – odbiorniki te składają się z określonej liczby niezależnych kanałów i każdy z nich jest przystosowany do odbierania i przetwarzania sygnałów z jednego satelity. Procesy odbioru i przetwarzania sygnałów są prowadzone w takim wielokanałowym odbiorniku jednocześnie. Obserwacje mogą być wykonywane z częstotliwością sekundową.
\item multi plexing – odbiorniki te składają się z jednego lub wielu kanałów, z których każdy może odbierać poszczególne sygnały z satelitów. Obserwacje wykonywane są z częstotliwością milisekundową. Najlepszą jakość sygnału mają odbiorniki typu multi-channel correlation type. Odbiorniki squaring type kwadratują zarówno sygnały, jak i szumy.
\end{itemize}
\subsection{Protokół NMEA}
NMEA 0183 (krótko nazywany również NMEA) – opublikowany przez National Marine Electronics Association protokół komunikacji między morskimi urządzeniami elektronicznymi. Ma on powszechne zastosowanie w elektronice nawigacji morskiej oraz urządzeniach GPS
Dane są transmitowane w postaci „zdań” zapisanych kodem ASCII. Pojedyncza sekwencja zawiera do 82 znaków. Znakiem zaczynającym dane w protokole jest „\$”, dalej następuje identyfikator zdania i pola danych oddzielone przecinkami, a na końcu znajdują się symbole <CR><LF> (carriage return, line feed).
\section{Kod zródłowy}

\begin{verbatim}
	using System;
	using System.Windows.Forms;
	using System.IO.Ports;
	
	namespace GPS_BB
	{
		
		public partial class Form1 : Form
		{
			public Form1()
			{
				InitializeComponent();
			}
						//utworzenie portu
			SerialPort port = new SerialPort("COM3", 9600, Parity.None, 8, StopBits.One);
			private string data = "";
			string receivedData;
			delegate void A();
			A delegat;
			private void button1_Click(object sender, EventArgs e)
			{
						//otwarcie portu
				port.DataReceived += new SerialDataReceivedEventHandler(DataReceivedEventHandler);
				port.Open();
				if (port.IsOpen)
				MessageBox.Show("Port otwarty");
				
				else
				MessageBox.Show("Błąd otwierania portu");
			}
			private void DataRecievedHandler(object sender, SerialDataReceivedEventArgs e)
			{
				textBoxReceivedFiles.Invoke(delegat);
			}
			
			private void WpiszOdebrane()
			{
				textBoxReceivedFiles.Text += port.ReadByte().ToString("X") + " ";
			}
			
			public void DataReceivedEventHandler(object sender, SerialDataReceivedEventArgs e)
			{
				if (InvokeRequired)
				; // Invoke(DataReceivedEventHandler(sender,e), sender, e);
				else
				{
					var senderPort = (SerialPort)sender;
					data += senderPort.ReadExisting();
					textBox1.Text = data;
				}
			}
			
			private void button2_Click(object sender, EventArgs e)
			{
					//zamykanie portu oraz zczytanie danych z GPSa
				port.Close();
				var ns = 0.0;
				var we = 0.0;
				string[] tempStr = null;
				bool gpsSignal = false;
				foreach (var line in textBoxReceivedFiles.Lines)
				{
					if (line.Contains("GPGGA"))
					{
						tempStr = line.Split(',');
						if (tempStr[2] != "")
						{
							gpsSignal = true;
							ns = Convert.ToDouble(tempStr[2]) / 100;
							we = Convert.ToDouble(tempStr[4]) / 100;
							
						}
						
					}
					
				}
				if (!gpsSignal)
				{
				//	MessageBox.Show("Błąd");
				}
				else
				{
				//pobranie współrzędnych oraz wyświtlenie pozycji
					if (tempStr[3] == "S")
					ns *= -1;
					if (tempStr[5] == "W")
					we *= -1;
					var link = $"https://www.google.pl/maps/@{ns},{we},14z";
					webBrowser1.Navigate(link);
					textBoxReceivedFiles.Text = link;
				}
				port.DataReceived += new SerialDataReceivedEventHandler(DataRecievedHandler);
				delegat = new A(WpiszOdebrane);
			}
			
			private void textBoxReceivedFiles_TextChanged(object sender, EventArgs e)
			{
				textBoxReceivedFiles.SelectionStart = textBoxReceivedFiles.Text.Length;
				textBoxReceivedFiles.ScrollToCaret();
			}
		}
	}
	
	
\end{verbatim}
\end{document}